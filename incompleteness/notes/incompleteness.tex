\documentclass[12pt]{article}

\usepackage{mathtools}

\usepackage{amsmath}
\usepackage{amssymb}
\usepackage{amsthm}

\usepackage{tcolorbox}

\DeclarePairedDelimiter{\enc}{[\![}{]\!]}

\newcommand{\prov}{\text{Prov}}
\newcommand{\PA}{\texttt{PA}}

\newtheorem{theorem}{Theorem}

\title{Incompleteness\\\large C\&! 2025}
\author{Daniel Hader}
\date{}

\begin{document}
	\maketitle
	
	\section{First order logic}
	
	First order logic is, in a very real sense, a programming language for mathematics
	
	\subsection{Universe of discourse}
	\subsection{Model theory}
	\subsubsection{Predicates}
	\subsubsection{Functions}	
	\subsubsection{Relations}
	
	Equality axioms
	
	\begin{itemize}
		\item $\forall x (x = x)$
		\item $\forall x, y (x=y \to y=x)$
		\item $\forall x, y, z (((x=y) \land (y=z)) \to (z=z))$
	\end{itemize}
	
	\section{Peano Arithmetic}
	
	\subsection{Axioms of Peano arithmetic}
	In addition to the axioms of FOL, Peano arithmetic introduces:
	
	The signature of \PA includes the following symbols:
	\begin{itemize}
		\item $0$ a constant or 0-ary function, to be interpreted as zero,
		\item $S$ a 1-ary function, to be interpreted as succession,
		\item $+$ a 2-ary function, to be interpreted as addition, and
		\item $\cdot$ a 2-ary function, to be interpreted as multiplication,
		\item $\le$ a 2-ary relation, to be interpreted as no greater than,
	\end{itemize}
	
	Notice that we typically define $+$ and $\cdot$ to be written in infix notation, that is in-between its operands, but we could just as well write $+(x, y)$ or $\cdot(x, y)$ to mean the same thing if you're worried about a more concise grammar.
	
	Additionally PA includes the following axioms
	
	\begin{tcolorbox}[colback=green!5!white,colframe=green!75!black,title={\bf Successor axioms}]
		\begin{itemize}
			\item $\forall x, y ((S(x)=S(y)) \to (x=y))$
			\item $\forall x (\lnot (S(x)=0))$ 
		\end{itemize}
	\end{tcolorbox}
	
	\begin{tcolorbox}[colback=green!5!white,colframe=green!75!black,title={\bf Addition axioms}]
		\begin{itemize}
			\item $\forall x (x+0 = x)$
			\item $\forall x, y (x+S(y) = S(x+y))$
			\item $\forall x, y (x+y=y+x)$ (redundant)
		\end{itemize}
	\end{tcolorbox}
	
	\begin{tcolorbox}[colback=green!5!white,colframe=green!75!black,title={\bf Multiplication axioms}]
		\begin{itemize}
			\item $\forall x (x+0 = x)$
			\item $\forall x, y (x+S(y) = S(x+y))$
			\item $\forall x, y (x+y=y+x)$ (redundant)
		\end{itemize}
	\end{tcolorbox}
	
	\begin{tcolorbox}[colback=green!5!white,colframe=green!75!black,title={\bf Inequality axioms}]
		TODO
	\end{tcolorbox}
	
		
	\section{Encodings}
	\section{The provability predicate}
	\section{The Incompleteness Theorems}
	
	\begin{theorem}[The first incompleteness theorem]
		There exists a sentence $S$ in the language of Peano arithmetic with neither $S$ nor $\lnot S$ provable in PA.
	\end{theorem}

	\begin{theorem}[The second incompleteness theorem]
		Let $C$ be the sentence $\lnot\prov(\enc{0=1})$, then neither $C$ nor $\lnot C$ has a proof in PA.		
	\end{theorem}

	\begin{tcolorbox}[colback=red!5!white,colframe=red!75!black,title=TODO]
		intuitive descriptions
	\end{tcolorbox}

	The above theorems are true in more general cases, any superset of PA, but essentially any theory capable of constructing the provability predicate is sufficient.
	
\end{document}