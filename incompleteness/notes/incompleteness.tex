\documentclass[12pt]{article}

\usepackage{mathtools}
\usepackage{enumitem}
\usepackage{amsmath}
\usepackage{amssymb}

\usepackage{tcolorbox}
\tcbuselibrary{theorems}

\newtcbtheorem{theorem}{Theorem}{colback=blue!5!white,colframe=blue!60!white,fonttitle=\bfseries}{thm}
\newtcbtheorem{definition}{Definition}{colback=black!5!white,colframe=black!60!white,fonttitle=\bfseries}{def}


\DeclarePairedDelimiter{\enc}{[\![}{]\!]}

\newcommand{\prov}{\text{Prov}}
\newcommand{\PA}{\texttt{PA}}

\title{Incompleteness\\\large C\&! 2025}
\author{Daniel Hader}
\date{}

\begin{document}
	\maketitle
	
	\section{First order logic}
	
	% Add the fluff later
	
	First order logic, FOL from here, is a \emph{formal system} where \emph{formal} refers to symbolic form, not some notion of rigor.
	
	In FOL we describe a \emph{theory} using \emph{axioms} given in some precise language. Proof theory and model theory offer two different approaches to understanding these theories. In model theory, one tries to construct explicit mathematical objects that are described by the theory and hence which satisfy the axioms. In proof theory, the axioms are used as clues to reason about the objects in the theory more generally. 
	
	\subsection{Languages and Theories}
	
	\begin{definition}{Formal language}
		A \emph{formal language} is simply a collection of words made up from a fixed set of symbols called the \emph{alphabet}. 
	\end{definition}
	
	\noindent Examples include:
	\begin{itemize}[itemsep=0pt]
		\item the primes in base 10 $\{2, 3, 5, 7, 11, 13, \ldots \}$ over $\{0,1,2,3,4,5,6,7,8,9\}$
		\item the whole numbers written in English $\{\text{one},\text{two},\text{three},\text{four},\ldots\}$
		\item All Minecraft seeds that cause the player to spawn in the desert
	\end{itemize}
	
	\noindent The language of FOL includes the following symbols
	
	\noindent In FOL we make \emph{formulas} using 
	
	\subsubsection{Domain of discourse}
	
	A theory
	
	\subsubsection{Predicates}
	
	Predicates describe \emph{yes} or \emph{no} questions that we might ask about an object in our theory
	
	\subsubsection{Functions}	
	\subsubsection{Relations}
	
	Equality axioms
	
	\begin{itemize}
		\item $\forall x (x = x)$
		\item $\forall x, y (x=y \to y=x)$
		\item $\forall x, y, z (((x=y) \land (y=z)) \to (z=z))$
	\end{itemize}
	
	\section{Peano Arithmetic}
	
	\subsection{Axioms of Peano arithmetic}
	In addition to the axioms of FOL, Peano arithmetic introduces:
	
	The signature of \PA includes the following symbols:
	\begin{itemize}
		\item $0$ a constant or 0-ary function, to be interpreted as zero,
		\item $S$ a 1-ary function, to be interpreted as succession,
		\item $+$ a 2-ary function, to be interpreted as addition, and
		\item $\cdot$ a 2-ary function, to be interpreted as multiplication,
		\item $\le$ a 2-ary relation, to be interpreted as no greater than,
	\end{itemize}
	
	Notice that we typically define $+$ and $\cdot$ to be written in infix notation, that is in-between its operands, but we could just as well write $+(x, y)$ or $\cdot(x, y)$ to mean the same thing if you're worried about a more concise grammar.
	
	Additionally PA includes the following axioms
	
	\begin{tcolorbox}[colback=green!5!white,colframe=green!75!black,title={\bf Successor axioms}]
		\begin{itemize}
			\item $\forall x, y ((S(x)=S(y)) \to (x=y))$
			\item $\forall x (\lnot (S(x)=0))$ 
		\end{itemize}
	\end{tcolorbox}
	
	\begin{tcolorbox}[colback=green!5!white,colframe=green!75!black,title={\bf Addition axioms}]
		\begin{itemize}
			\item $\forall x (x+0 = x)$
			\item $\forall x, y (x+S(y) = S(x+y))$
			\item $\forall x, y (x+y=y+x)$ (redundant)
		\end{itemize}
	\end{tcolorbox}
	
	\begin{tcolorbox}[colback=green!5!white,colframe=green!75!black,title={\bf Multiplication axioms}]
		\begin{itemize}
			\item $\forall x (x+0 = x)$
			\item $\forall x, y (x+S(y) = S(x+y))$
			\item $\forall x, y (x+y=y+x)$ (redundant)
		\end{itemize}
	\end{tcolorbox}
	
	\begin{tcolorbox}[colback=green!5!white,colframe=green!75!black,title={\bf Inequality axioms}]
		TODO
	\end{tcolorbox}
	
		
	\section{Encodings}
	\section{The provability predicate}
	\section{The Incompleteness Theorems}
	
	\begin{theorem}{First incompleteness theorem}
		There exists a sentence $S$ in the language of Peano arithmetic with neither $S$ nor $\lnot S$ provable in PA.
	\end{theorem}

	\begin{theorem}{Second incompleteness theorem}
		Let $C$ be the sentence $\lnot\prov(\enc{0=1})$, then neither $C$ nor $\lnot C$ has a proof in PA.		
	\end{theorem}

	\begin{tcolorbox}[colback=red!5!white,colframe=red!75!black,title=TODO]
		intuitive descriptions
	\end{tcolorbox}

	The above theorems are true in more general cases, any superset of PA, but essentially any theory capable of constructing the provability predicate is sufficient.
	
\end{document}