\documentclass[10pt]{article}

\usepackage{fitch}
\usepackage{bussproofs}

\usepackage{mathtools}
\usepackage{enumitem}
\usepackage{amsthm}
\usepackage{amsmath}
\usepackage{amssymb}
\usepackage{todonotes}

\usepackage{tcolorbox}
\tcbuselibrary{theorems}

\newtcbtheorem{theorem}{Theorem}{colback=blue!5!white,colframe=blue!60!white,fonttitle=\bfseries}{thm}
\newtcbtheorem{definition}{Definition}{colback=black!5!white,colframe=black!60!white,fonttitle=\bfseries}{def}

\DeclareMathSymbol{:}{\mathord}{operators}{"3A}
\DeclarePairedDelimiter{\enc}{[\![}{]\!]}

\newcommand{\type}[1]{\texttt{#1}}
\newcommand{\prov}{\text{Prov}}
\newcommand{\PA}{\texttt{PA}}

\newcommand{\N}{\mathbb{N}}

\title{Incompleteness\\\large C\&! 2025}
\author{Daniel Hader}
\date{}

\begin{document}
	\maketitle
	
	\section{First order logic}
	
	% Add the fluff later
	
	First order logic, FOL from here, is a \emph{formal system} where \emph{formal} refers to symbolic form, not some notion of rigor.
	
	In FOL we describe a \emph{theory} using \emph{axioms} given in some precise language. Proof theory and model theory offer two different approaches to understanding these theories. In model theory, one tries to construct explicit mathematical objects that are described by the theory and hence which satisfy the axioms. In proof theory, the axioms are used as clues to reason about the objects in the theory more generally. 
	
	\todo{define or explain theory}
	
	\subsection{Terms}
	
	\begin{definition}{Domain of Discourse}{}
		A {\bf domain} or {\bf universe} of discourse is the collection of mathematical objects our theory is concerned with.
	\end{definition}
	
	\noindent
	Common universes include $\N$ the natural numbers in theories of arithmetic (the domain we will be concerned with here) and the universe of all sets in set theories.
	More complex theories may have different types of objects in their domains.
	For instance a theory might contain both vector spaces and pairs of vector spaces as objects of concern.
	
	\begin{tcolorbox}[colback=red!5!white,colframe=red!75!black,title={\bf Hope someone asks about it}]
		Why aren't domains the same things as sets?
	\end{tcolorbox}
	
	
	\begin{definition}{Variable symbols}{}
		A {\bf variable symbol} is any of a collection of symbols used to abstractly represent an element of our domain of discourse.
	\end{definition}
	
	\noindent 
	Variable symbols might include the likes of $x$, $y$, or $z$, along with sub- or super-scripted versions like $x_{100}$ or $y^n$, so long as we make it clear which symbols are variable symbols.
	We may imagine that we have as many variables as we need or infinitely many, which ever is conceptually easier.
	
	\begin{definition}{Function symbols}{}
		A {\bf $k$-ary function symbol} is any of a collection of symbols used to abstractly represent a function from $k$ objects in the universe to one object in the universe.
	\end{definition}

	\noindent
	The \emph{arity} of a function is how many things it applies to.
	Together, these allow us to define terms.
	
	\begin{definition}{Terms}{}
		A {\bf term} is one of the following:
		\begin{itemize}[noitemsep,topsep=0pt]
			\item a variable symbol $v:V$, or
			\item the symbolic application $f(t_1,\ldots,t_n)$ of an $n$-ary function symbol $f:\type{F}^n$ to terms $t_1:\type{T},\ldots,t_n:\type{T}$
		\end{itemize}
	\end{definition}

	\noindent
	A term is intended to represent an object in the domain of discourse.
	Notice that symbolic function application is a \emph{formal} operation denoted by a function symbol $f$ before a term or comma separated list of terms surrounded by parentheses.
	This notation is more or less arbitrary.
	
	\subsection{Formula}
	
	\begin{definition}{Predicate symbols}{}
		An {\bf $n$-ary predicate symbol}, denoted by the type $\type{P}^n$, represents  statement about terms.
	\end{definition}

	\noindent
	Examples include predicates asking if a natural number is prime.
	
	The next important concept is that of \emph{formulas}.
	These are defined inductively, like terms, from simple forms of other objects and smaller formula.
	
	\begin{definition}{Formulas}{}
		A {\bf forumla} of arity-$k$ is denoted with the type specifier $\type{F}^k$ is anything which matches any of the following patterns:
		\begin{itemize}[itemsep=0pt]
			\item The symbolic application $P(t_1,\ldots,t_n)$ of an $n$-ary predicate symbol $P:\type{P}^n$ to terms $t_1:\type{T},\ldots,t_n:\type{T}$
		\end{itemize}
	\end{definition}
	
	
	Equality axioms
	
	\begin{tcolorbox}[colback=green!5!white,colframe=green!75!black,title={\bf Axioms of equality}]
		\begin{enumerate}[itemsep=0pt, label={E.\arabic*}]
			\item $\forall x. (x = x)$ \label{ax:eq-refl}
			\item $\forall x. \forall y. (x=y \to y=x)$ \label{ax:eq-sym}
			\item $\forall x. \forall y. \forall z. (((x=y) \land (y=z)) \to (z=z))$ \label{ax:eq-trans} 
			\item $\forall x. \forall y. (x = y) \to (\phi(\dots,x,\ldots) \to \phi(\ldots,y,\ldots))$ \hfill $\phi:F$
		\end{enumerate}
	\end{tcolorbox}

	\noindent
	
	For the last equality axiom, we really would like to say something like:
	$$\forall \phi. \forall x. \forall y. (x = y) \to (\phi(\dots,x,\ldots) \to \phi(\ldots,y,\ldots))$$
	but this is impossible in FOL since quantifiers only act over objects in our universe.
	Higher order logics give you some ability to do this, but in FOL we can skirt this problem by using an \emph{axiom schema}.
	We interpret the last axiom. not as 1 axiom, but rather an infinite list of axioms, one for each formula $\phi$ with at least 1 free variable and where the $x$ and $y$ are plugged into every possible free variable of $\phi$.
	
	It doesn't particularly matter that we couldn't technically write all of the axioms.
	Instead it's okay so long as there's an algorithmic way to construct the sequence of characters defining the axiom.
	 
	\section{Peano Arithmetic}
	
	\subsection{Axioms of Peano arithmetic}
	In addition to the axioms of FOL, Peano arithmetic introduces:
	
	The signature of \PA includes the following symbols:
	\begin{itemize}
		\item $0$ a constant or 0-ary function, to be interpreted as zero,
		\item $S$ a 1-ary function, to be interpreted as succession,
		\item $+$ a 2-ary function, to be interpreted as addition, and
		\item $\cdot$ a 2-ary function, to be interpreted as multiplication,
		\item $\le$ a 2-ary relation, to be interpreted as no greater than,
	\end{itemize}
	
	Notice that we typically define $+$ and $\cdot$ to be written in infix notation, that is in-between its operands, but we could just as well write $+(x, y)$ or $\cdot(x, y)$ to mean the same thing if you're worried about a more concise grammar.
	
	Additionally PA includes the following axioms
	
	\begin{tcolorbox}[colback=green!5!white,colframe=green!75!black,title={\bf Successor axioms}]
		\begin{enumerate}[label=S.\arabic*]
			\item $\forall x, y ((S(x)=S(y)) \to (x=y))$ \label{ax:succ-inj}
			\item $\forall x (\lnot (S(x)=0))$ \label{ax:succ-first}
		\end{enumerate}
	\end{tcolorbox}
	
	\begin{tcolorbox}[colback=green!5!white,colframe=green!75!black,title={\bf Addition axioms}]
		\begin{enumerate}[label=A.\arabic*]
			\item $\forall x (x+0 = x)$ \label{ax:add-id}
			\item $\forall x, y (x+S(y) = S(x+y))$ \label{ax:add-succ}
			\item $\forall x, y (x+y=y+x)$ (redundant) \label{ax:add-sym}
		\end{enumerate}
	\end{tcolorbox}
	
	\begin{tcolorbox}[colback=green!5!white,colframe=green!75!black,title={\bf Multiplication axioms}]
		
	\end{tcolorbox}
	
	\begin{tcolorbox}[colback=green!5!white,colframe=green!75!black,title={\bf Inequality axioms}]
		TODO
	\end{tcolorbox}
	
	\section{Proof system}
	
	Here we introduce a proof system called \emph{natural deduction} which is designed to formally mimic the form of natural language proofs in mathematics.
	
	Let's try and prove the following:
	
	\begin{theorem}{}{}
		$1 + 1 = 2$, or in the language of PA
		
		$$S(0) + S(0) = S(S(0))$$
	\end{theorem}
	
	\noindent
	First we have to formalize this statement in the language of PA:
	
	\[\begin{nd}
		\have {1} {\forall x. \forall y. (x + S(y) = S(x+y))} \by{Axiom~\ref{ax:add-succ}}{}
		\have {2} {\forall y. (S(0) + S(y) = S(S(0) + y)} \by{$\forall_E [x \mapsto S(0)]$}{1}
		\have {3} {S(0) + S(0) = S(S(0)+0)}\by{$\forall_E [y \mapsto 0]$}{2}
		\have {4} {\forall x. (x + 0 = x)} \by{Axiom~\ref{ax:add-id}}{}
		\have {5} {S(0)+0 = S(0)}\by{$\forall_E [x \mapsto S(0)]$}{4}
		\have {6} {}
		
	\end{nd}\]
	
	\section{Encodings}
	\section{The provability predicate}
	
	\begin{definition}{The provability predicate}{}
		The provability predicate $\prov(n)$ is defined to be true when $n$ is the encoding $\enc{S}$ of some provable sentence $S$ in PA, and false otherwise.
	\end{definition}
	
	It's not at all clear that the provability predicate can even be defined in PA, but it can be owing to the power of encoding.
	
	\section{The Incompleteness Theorems}
	
	\begin{theorem}{First incompleteness theorem}{}
		There exists a sentence $S$ in the language of Peano arithmetic with neither $S$ nor $\lnot S$ provable in PA.
	\end{theorem}

	The proof of this theorem is very simple:
	
	\begin{proof}
		Let $R$ be a 2-ary relation defined so that $R(n,m)$ is true if and only if all of the following are satisfied:
		\begin{enumerate}[itemsep=0pt]
			\item $m$ is the encoding of some formula $F(x)$ with a single free variable $x$
			\item $n$ is the encoding of a proof of $F(m)$
		\end{enumerate}
		
		\noindent Now let $G$ be the formula $G(x) = \forall y (\lnot R(y, x))$ where $x$ is a free variable.
		
		The godel sentence is $G(\enc{G})$ 
	\end{proof}

	\begin{theorem}{Second incompleteness theorem}{}
		Let $C$ be the sentence $\lnot\prov(\enc{0=1})$, then neither $C$ nor $\lnot C$ has a proof in PA.		
	\end{theorem}

	\begin{tcolorbox}[colback=red!5!white,colframe=red!75!black,title=TODO]
		intuitive descriptions
	\end{tcolorbox}

	The above theorems are true in more general cases, any superset of PA, but essentially any theory capable of constructing the provability predicate is sufficient.
	
\end{document}