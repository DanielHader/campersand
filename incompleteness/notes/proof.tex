
In this section we discuss a proof system called \textbf{natural deduction} and explore it in the context of both propositional and first order logic.
The following is a proof of a statement in propositional logic using natural deduction in the notation of Suppes and Lemmon.
Consider its parts, try to understand them, and compare them to a what you think a good written proof should look like.

\medskip

\begin{center}
	\begin{lproof}{Proof: $(A\to(B\to C)) \to ((A\land B)\to C)$}
		\pline{1}{Assume}{1}{A\to(B\to C)}{}
		\pline{2}{Assume}{2}{A\land B}{}
		\pline{3}{$\land$Elim$_1$ 2}{2}{A}{}
		\pline{4}{$\land$Elim$_2$ 2}{2}{B}{}
		\pline{5}{$\to$Elim 3,1}{1,2}{B\to C}{}
		\pline{6}{$\to$Elim 4,5}{1,2}{C}{}
		\pline{7}{$\to$Intro 2,6}{1}{(A\land B)\to C}{}
		\hline
		\pline{8}{$\to$Intro 1,7}{}{(A\to(B\to C)) \to ((A\land B)\to C)}{}
	\end{lproof}
\end{center}

\subsection{Propositional Logic}

In propositional logic, we are concerned with {\bf propositions}, statements about the world like ``Today is Wednesday'' or ``The Intel 8080 CPU was released in 1974''.
These may be combined using {\bf connectives} to yield {\bf formula}.

\begin{definition}{Propositional Formula}{}
	In propositional logic, a {\bf formula} matches one of the following patterns:
	\begin{itemize}[noitemsep, topsep=0pt]
		\item a proposition $A:\type{Prop}$,
		\item the {\bf negation} of a formula $\lnot (A:\type{Form})$,
		\item the {\bf conjunction} of two formula $(A:\type{Form}) \land (B:\type{Form})$,
		\item the {\bf disjunction} of two formula $(A:\type{Form}) \lor (B:\type{Form})$,
		\item an {\bf implication} between two formula $(A:\type{Form})\to(B:\type{Form})$,	
\end{itemize}
\end{definition}

Additionally, we define a special symbol

\begin{definition}{Top and bottom}{}
	The special symbol $\bot$, sometimes called {\bf bottom} or {\bf false}, denotes a contradiction.
\end{definition}

There is a corresponding symbol $\top$, called {\bf top} or {\bf true}, denoting a statement that is tautologically true.
This symbol is much less useful than $\bot$ in propositional logic, we probably wont see it much more. Also, not all authors even choose to include $\bot$ or $\top$ as symbols in propositional logic, they're an optional feature.
It's analogous to the concept of {\bf syntactic sugar} in programming.
We can avoid $\bot$ ($\top$) all together by replacing any instance of $\bot$ ($\top$) with an explicitly contradictory (tautological) formula like $A\land \lnot A$ ($A \lor \lnot A$).
The resulting logic system is identical.

\begin{discuss}{}{}
	What makes $A \land \lnot A$ a contradictory formula? What about $A \lor \lnot A$?
	What does this say about the nature of a proposition's truth value? 
\end{discuss}

\todo{discuss explosion at some point} 

\subsection{Natural Deduction}

Now that we understand the notation of propositional logic, lets explore the rules for writing {\bf proofs}.
In propositional logic, and indeed in most any logic, proof systems are understand to be symbolic pattern matching games whose rules somehow reflect valid modes of reasoning.
In other words, what one is allowed to write in each line of a proof is constrained by a set of rules to enforce logical validity. The following rules are to be understood as pattern templates.
\todo{explain this dork}

\paragraph{Assumption} is the simplest rule, you simply assume a proposition as a premise.

\begin{tcolorbox}[colback=purple!5!white,colframe=purple!75!black,title={\bf Conjunction Rules}]
	\centering
	\begin{lproof}{Assumption}
		\hline
		\pline{$i$}{Assume}{$i$}{A:\type{Form}}{}
	\end{lproof}
\end{tcolorbox}

From nothing this proof rule allows you to assume any formula $A$, but notice how the line number of the assumption was added to the {\bf assumption list}.
Any line with a formula constructed using this assumption will inherit that entry in their assumption list with one major exception involving implication that will be discussed soon.

\paragraph{Conjunction} or the ``and'' connective is defined by the following rules which formally represent exactly what you'd expect of an symbol for ``and''.

\begin{tcolorbox}[colback=purple!5!white,colframe=purple!75!black,title={\bf Conjunction Rules}]
	\centering
	\begin{lproof}{And introduction}
		\pline{$i$}{}{$i_1,\ldots,i_m$}{A:\type{Form}}{}
		\pline{$j$}{}{$j_1,\ldots,j_n$}{B:\type{Form}}{}
		\hline
		\pline{$k$}{$\land$Intro $i,j$}{$i_1,\ldots,i_m,j_1,\ldots,j_n$}{(A\land B):\type{Form}}{}
	\end{lproof}
	
	\tcblower
	
	\begin{lproof}{And elimination left}
		\pline{$i$}{}{$i_1,\ldots,i_m$}{(A\land B):\type{Form}}{}
		\hline
		\pline{$j$}{$\land$Elim$_1$ $i$}{$i_1,\ldots,i_m$}{A:\type{Form}}{}
	\end{lproof}
	\hfill
	\begin{lproof}{And elimination right}
		\pline{$i$}{}{$i_1,\ldots,i_m$}{(A\land B):\type{Form}}{}
		\hline
		\pline{$j$}{$\land$Elim$_2$ $i$}{$i_1,\ldots,i_m$}{B:\type{Form}}{}
	\end{lproof}
\end{tcolorbox}

\paragraph{Disjunction} or the ``or'' connective is defined similarly.
Compare these rules to those of conjunction.
If you're bothered by the awkwardness of the elimination rule, there are alternative proof systems like Gentzen's sequent calculus whose rules are more symmetric at the cost of some additional complexity.

\begin{tcolorbox}[colback=purple!5!white,colframe=purple!75!black,title={\bf Disjunction Rules}]
	\begin{lproof}{Or introduction left}
		\pline{$i$}{}{$i_1,\ldots,i_m$}{A:\type{F}}{}
		\hline
		\pline{$j$}{$\lor$Intro$_1$ $i$}{$i_1,\ldots,i_m$}{(A\lor B):\type{F}}{}
	\end{lproof}
	\hfill
	\begin{lproof}{Or introduction right}
		\pline{$i$}{}{$i_1,\ldots,i_m$}{B:\type{F}}{}
		\hline
		\pline{$j$}{$\lor$Intro$_2$ $i$}{$i_1,\ldots,i_m$}{(A\lor B):\type{F}}{}
	\end{lproof}
	
	\tcblower
	
	\centering
	\begin{lproof}{Or elimination}
		\pline{$i$}{}{$i_1,\ldots,i_m$}{(A\lor B):\type{F}}{}
		\pline{$j$}{}{$j_1,\ldots,j_n$}{(A\to C):\type{F}}{}
		\pline{$k$}{}{$k_1,\ldots,k_o$}{(B\to C):\type{F}}{}
		\hline
		\pline{$l$}{$\lor$E}{$i_1,\ldots,i_m,j_1,\ldots,j_n,k_1,\ldots,k_o$}{C:\type{F}}{}
	\end{lproof}
\end{tcolorbox}

\begin{tcolorbox}[colback=purple!5!white,colframe=purple!75!black,title={\bf Implication Rules}]
	\begin{lproof}{Implication Introduction}
		\pline{$i$}{Assume}{}{A:\type{F}}{}
		\pline{$j$}{}{$i$}{B:\type{F}}{}
		\hline
		\pline{$k$}{$\to$I}{$i$, $j$}{A\to B:\type{F}}{}
	\end{lproof}
	\hfill
	\begin{lproof}{Implication Elimination}
		\pline{$i$}{}{}{A:\type{F}}{}
		\pline{$j$}{}{}{A\to B:\type{F}}{}
		\hline
		\pline{$k$}{$\to$E}{$i$, $j$}{B:\type{F}}{}
	\end{lproof}
\end{tcolorbox}

\begin{tcolorbox}[colback=purple!5!white,colframe=purple!75!black,title={\bf Negation Rules}]
	Negation is understood to mean implying a contradiction:
	$$\lnot A \equiv A\to\bot$$
	
	\tcblower
	
	\begin{lproof}{Double negation elimination}
		\pline{$i$}{}{}{\lnot\lnot A:\type{F}}{}
		\hline
		\pline{$j$}{DNE}{$i$}{A:\type{F}}{}
	\end{lproof}
	\hfill
	\begin{lproof}{Reductio ad absurdum}
		\pline{$i$}{}{}{A:\type{F}}{}
		\pline{$j$}{}{$i$}{\bot:\type{P}^0}
		\hline
		\pline{$k$}{RAA}{$i$, $j$}{\lnot A:\type{F}}{}
	\end{lproof}
\end{tcolorbox}

\begin{tcolorbox}[colback=purple!5!white,colframe=purple!75!black,title={\bf Negation Rules}]
	\begin{lproof}{Negation Introduction}
		\pline{$i$}{Assume}{}{A:\type{F}}{}
		\pline{$j$}{}{$i$}{B:\type{F}}{}
		\hline
		\pline{$k$}{$\to$I}{$i$, $j$}{A\to B:\type{F}}{}
	\end{lproof}
	\hfill
	\begin{lproof}{Implication Elimination}
		\pline{$i$}{}{}{A:\type{F}}{}
		\pline{$j$}{}{}{A\to B:\type{F}}{}
		\hline
		\pline{$k$}{$\to$E}{$i$, $j$}{B:\type{F}}{}
	\end{lproof}
\end{tcolorbox}